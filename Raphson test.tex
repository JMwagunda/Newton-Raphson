
\documentclass{article}
\usepackage{amsmath}

\begin{document}

In order to determine the function's root within a given tolerance and the maximum number of iterations, this function uses the Newton-Raphson approach. Initial guess x0, tolerance, and maximum iterations are provided as inputs. If the root cannot be located within the provided number of iterations, the function emits an error message instead of the root when it is discovered. Moreover, the function needs the derivative, which is represented by the symbol df (x).\\

Consider the function $f(x) = 2x^3 - 3^x - 12x$ and its derivative $f'(x) = 6x^2 - 6x - 12$.

Let $x_0 = 1$ be the initial guess for the root of $f(x)$.

The Newton-Raphson method is used to find the root of the function. The algorithm iteratively updates the value of $x_0$ using the following equation:

$$ x_1 = x_0 - \frac{f(x_0)}{f'(x_0)} $$

If the absolute value of $f(x_1)$ is less than the tolerance level, then the algorithm terminates and the root is found.

Otherwise, the value of $x_0$ is updated to $x_1$ and the process continues.

Python code's implementation of Newton-Raphson method 

\begin{verbatim}
import timeit

def f(x):
    return 2*x**3 - 3**x - 12*x

def df(x):
    return 6*x**2 - 6*x-12

x0 = 1
tolerance = 1e-6
max_iteration = 100

for i in range(max_iteration):
    fx = f(x0)
    dfx = df(x0)
    
    x1 = x0 - (fx / dfx)
    print("Iterations " + str(i) + ": x = " + str(x0) + " f(x) = " +    
          str(f(x0)))
    
    if abs(f(x1)) < tolerance:
        print(f"Root found at x = {x1:.2f}")
        print("Time taken is ", timeit.timeit())
        break
    else:
        x0 = x1
\end{verbatim}

The output of the Python code is printed to the console

This includes the iteration number, the root found, and the time taken to find the root.

\end{document}
